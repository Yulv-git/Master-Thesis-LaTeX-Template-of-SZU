% % 主文档,包含了 SZU Master Thesis LaTex 全篇的主要架构。

% % 载入模板类
% \documentclass{szuthesis}  % 默认形式
% \documentclass[print]{szuthesis}  % 打印预览版本,可自动生成额外的空白页用于打印
% \documentclass[fontset=windows|adobe|mac|ubuntu]{szuthesis}  % 选择字库
\documentclass[fontset=windows]{szuthesis}

% % 载入配置信息,包含论文封面信息、必要的package
\input{config}

% % 辅助命令,后文中的所有\include均可在此单独列出,用逗号隔开,以此只编译必要的章节,加快编译速度,
% \includeonly{Tex/Abstract,Tex/Appendix}  % 待全文完成后可注释本命令,即可编译全文。

\begin{document}

\maketitle  % 制作封面

% % 声明
% 如果参数为空则可自动生成默认声明页,也可设置参数导入签字后的扫描版PDF文件
\makedeclaration{}  % 制作声明页面,自动生成
% \makedeclaration{XX.pdf}  % 导入签名后的声明页扫描版,参数为pdf文档名,默认在Image下
% \szuaddpdf{XX.jpg}  % 若\makedeclaration{XX.pdf}导致摘要页排版异常,则使用图像格式的扫描文档导入

\frontmatter  % 初始化摘要页环境,不建议注释

% % 中、英文摘要

% % 中文摘要
\begin{abstract}
    ......

    \keywords{......,......,......}  % 中文关键词
\end{abstract}

% % 英文摘要
\begin{ABSTRACT}
    ...

    \KEYWORDS{..., ..., ...}  % 英文关键词
\end{ABSTRACT}


\tableofcontents  % 目录

\mainmatter  % 初始化正文环境,不建议注释

% % Introduction

\chapter{引言}

\section{研究背景与意义}

......\cite{cormen2022introduction}......。

.....\cite{lecun2015deep,李晓磊2002一种基于动物自治体的寻优模式}。
如图~\ref{fig:图的标签} 所示:......。

\begin{figure}[!htbp]
      \centering
      \subcaptionbox{......。}{\includegraphics[width=0.45\textwidth]{1_1.png}}
      \subcaptionbox{......。}{\includegraphics[width=0.45\textwidth]{1_2.png}}
      \caption{......。}\label{fig:图的标签}
\end{figure}

......

\section{国内外研究现状}

......

\section{研究挑战与目标}

......

\begin{enumerate}[wide=\parindent]
      \item ......
      \item ......
      \item ......
\end{enumerate}

......

\section{章节内容组织}

......


% % ChapterX.tex,论文的各个章节。

\chapter{22}

......

\section{第1节}

......

......公式~\ref{equ:公式}:
\begin{equation}\label{equ:公式}
    y=f(x), \quad x \in \symbb{R}^{n}, y \in \{0,1\},
\end{equation}
其中,......

......

\section{第2节}

......见图~\ref{fig:图像}。

\begin{figure}[!htbp]
    \centering
    \includegraphics[width=0.7\textwidth]{2.png}
    \caption{......。
    }\label{fig:图像}
\end{figure}

......

\section{第3节}

......

......请见表~\ref{tab:表格}。

\begin{table}[!htbp]
    \caption{......。}\label{tab:表格}
    \centering
    \footnotesize
    \setlength{\tabcolsep}{4pt}{
        \begin{tabular}{ccc}
            \hline
            c1    & c2  & c3        \\ \hline
            r1    & 111 & 2333      \\
            r2    & 666 & 12580     \\ \hline
        \end{tabular}}
\end{table}

......

\section{本章小结}

......


% % ChapterX.tex,论文的各个章节。

\chapter{333}

......

\section{第1节}

......

\subsection{第1.1节}

......

\subsection{第1.2节}

......

\subsection{第1.3节}

......

\section{第2节}

......

\section{第3节}

......

\section{本章小结}

......


% % ChapterX.tex,论文的各个章节。

\chapter{4444}

......

\section{第1节}

......

\section{第2节}

......

\section{第3节}

......

\section{本章小结}

......


% % 结论

\chapter[结论]{结\quad{}论}

\section{研究总结}

......

\section{主要贡献与创新}

......

\section{不足与展望}

......


\backmatter  % 初始化其他部分环境,不建议注释

\szubibliography  % 导入参考文献

% % 附录

\chapter[附录]{附\quad{}录}

\section*{附录1:......}

......
  % 导入附录

% % 添加答辩记录
% 建议分三个独立的PDF或图像。\szuaddpdf包含两个参数,[]中为可选参数,用于生成目录,{}中为文件名,默认在Image下
% \szuaddpdf[指导教师对研究生学位论文的学术评语]{pingyu.pdf}
% \szuaddpdf[学位论文答辩委员会决议书]{dabian1.pdf}
% \szuaddpdf{dabian2.pdf}  % 前一页生成目录即可

% % 致谢

\chapter[致谢]{致\qquad{}谢}

......
  % 导入致谢  % 盲审版,注释掉此处

% % 研究成果

\chapter{攻读硕士学位期间的研究成果}

%% 可直接使用引用的格式
\begin{enumerate}[label = {[\arabic*]}]
    \item Cormen T H, Leiserson C E, Rivest R L, et al. Introduction to algorithms[M]. MIT press, 2022.
    \item LeCun Y, Bengio Y, Hinton G. Deep learning[J]. nature, 2015, 521(7553): 436-444.
    \item 李晓磊, 邵之江, 钱积新. 一种基于动物自治体的寻优模式: 鱼群算法[J]. 系统工程理论与实践, 2002, 22(11): 32-38.
\end{enumerate}

%% 或者区分开论文和专利
% \section*{论文}
% \begin{enumerate}[label = {[\arabic*]}]
%     \item Cormen T H, Leiserson C E, Rivest R L, et al. Introduction to algorithms[M]. MIT press, 2022.
%     \item LeCun Y, Bengio Y, Hinton G. Deep learning[J]. nature, 2015, 521(7553): 436-444.
% \end{enumerate}
% \section*{专利}
% \begin{enumerate}[label = {[\arabic*]}]
%     \item 李晓磊, 邵之江, 钱积新. 一种基于动物自治体的寻优模式: 鱼群算法[J]. 系统工程理论与实践, 2002, 22(11): 32-38.
% \end{enumerate}
  % 导入研究成果  % 盲审版,注释掉此处

\end{document}

% 查重时,PaperYY等网站会把封面、原创性申明、附录、致谢等算进去。将摘要-参考文献部分的PDF拆分出来,再进行PaperYY等网站的查重。
% 若是使用PaperYY查重,可使用https://www.ilovepdf.com/zh-cn/split_pdf网页或者其他PDF拆分工具。
